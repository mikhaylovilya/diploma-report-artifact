\documentclass{vkr-slides-style}

\filltitle{
    % Ваше ФИО.
    author = {Василий Иванович Пупкин},
    % Нельзя оставлять пустые строки.
    % Ваше сокращённое именование, будет показываться слева внизу слайда.
    authorShort = {Василий Пупкин},
    %
    % Официальное название ВКР.
    title = {Полный официальный заголовок квалификационной работы},
    %
    % Короткое название ВКР, будет снизу по центру слайда.
    titleShort = {Короткое название},
    %
    % Научный руководитель.
    advisor = {к.ф.-м.н. А.А. Андреев, доцент кафедры системного программирования},
    %
    % Консультант (если есть). Если нет, оставьте пустые фигурные скобки.
    consultant = {П.П. Петров, программист ЗАО «Компания с ну очень-очень-очень длинным названием»},
    %
    % Дата доклада.
    date = {20.11.2023}
}
 
\begin{document}

\makeslidestitle

\begin{frame}  
    \frametitle{Суть работы}
    \begin{itemize}
        \item Тут кратко и по делу рассказываете, что именно хотите сделать
        \item Обозначьте отчуждаемый результат (что это будет --- приложение, инструмент, библиотека, ...), открыт ли код
        \item Должно быть понятно, в чём решаемая проблема, почему нельзя взять готовое решение, и кому это нужно (желательно с точностью до конкретной компании, а не вообще --- если работодатель предложил или утвердил тему, укажите это тут)
        \item Должен быть понятен объём работы (почему это диплом)
        \item Должно быть понятно, при чём тут СП\footnote{Ещё не поздно защищаться на ИАС или информатике.}
    \end{itemize}
\end{frame}

\begin{frame}  
    \frametitle{Постановка задачи}
    \textbf{Цель:} Краткая и ёмкая формулировка цели

    \vspace{5mm}
    \textbf{Задачи:}
    \begin{enumerate}
        \item Выполнить обзор того-то
        \item Спроектировать то-то
        \item Реализовать то-то
        \item Выполнить апробацию того-то
    \end{enumerate}
\end{frame}

\begin{frame}  
    \frametitle{План работы}
    Тут уместно рассказать подробности по каждому пункту, но кратко.
    
    \textbf{Что уже сделано:}
    \begin{enumerate}
        \item Выполнить обзор того-то
    \end{enumerate}

    \textbf{Планируется к зимней защите:}
    \begin{enumerate}
        \setcounter{enumi}{1}
        \item Спроектировать то-то
        \item Реализовать то-то
    \end{enumerate}

    \textbf{Планируется к защите ВКР:}
    \begin{enumerate}
        \setcounter{enumi}{3}
        \item Выполнить апробацию того-то
    \end{enumerate}
\end{frame}

\end{document}