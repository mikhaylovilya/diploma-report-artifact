% !TeX spellcheck = ru_RU
% !TeX spellcheck = en_US
\documentclass{vkr-slides-style}

\filltitle{
    % Ваше ФИО.
    author = {Михайлов Илья Игоревич},
    % Нельзя оставлять пустые строки.
    % Ваше сокращённое именование, будет показываться слева внизу слайда.
    authorShort = {Михайлов Илья},
    %
    % Официальное название ВКР.
    title = {Динамическая бинарная трансляция для архитектуры RISC-V с помощью Instrew},
    %
    % Короткое название ВКР, будет снизу по центру слайда.
    titleShort = {RISC-V Instrew},
    %
    % Научный руководитель.
    advisor = {к.ф.-м.н. Д.В. Луцив, доцент кафедры системного программирования},
    %
    % Консультант (если есть). Если нет, оставьте пустые фигурные скобки.
    consultant = {В.А. Кутуев, инженер-исследователь, Лаборатория технологий программирования инфраструктурных решений СПбГУ
    % , программист ЗАО «Компания с ну очень-очень-очень длинным названием»
    },
    %
    % Дата доклада.
    date = {23.11.2023}
}

\begin{document}

\makeslidestitle

\begin{frame}
    \frametitle{Суть работы}
    \begin{itemize}
        %     \item Тут кратко и по делу рассказываете, что именно хотите сделать
        %     \item Обозначьте отчуждаемый результат (что это будет --- приложение, инструмент, библиотека, ...), открыт ли код
        %     \item Должно быть понятно, в чём решаемая проблема, почему нельзя взять готовое решение, и кому это нужно (желательно с точностью до конкретной компании, а не вообще --- если работодатель предложил или утвердил тему, укажите это тут)
        %     \item Должен быть понятен объём работы (почему это диплом)
        %     \item Должно быть понятно, при чём тут СП\footnote{Ещё не поздно защищаться на ИАС или информатике.}
        \item Бинарные трансляторы являются популярными инструментами для анализа кода, профиляции и эмуляции, важным классом которых являются динамические бинарные трансляторы (далее --- ДБТ)
              % Бинарная трансляция -- это актуально, но бинарная трансляция с лифтингом двоичного кода в LLVM IR -- это еще и круто.
        \item В качестве промежуточного представления для ДБТ хорошо использовать LLVM IR
              \begin{itemize}
                  \item Тулчейн LLVM позволяет проводить качественные оптимизации
              \end{itemize}
              %Хотим добавить поддержку архитектуры RISC-V для такого ДБТ c поднятием в LLVM IR
        \item Instrew является ДБТ с поднятием кода в LLVM IR, но нет поддержки RISC-V, как хост-архитектуры
        \item Результатом работы является добавление поддержки архитектуры RISC-V для ДБТ Instrew с открытым исходным кодом
        \item В связи с наличием процессорно-специфических и низкоуровневых деталей реализации, задача является довольно трудоемкой
              % \item (Что по сути является классической задачей программной инженерии)
    \end{itemize}
\end{frame}

\begin{frame}
    \frametitle{Постановка задачи}
    \textbf{Цель:} Добавить поддержку архитектуры RISC-V для ДБТ Instrew.

    \vspace{5mm}
    \textbf{Задачи:}
    \begin{enumerate}
        \item Выполнить обзор архитектуры Instrew и сравнить с альтернативными ДБТ
              % \item Спроектировать то-то
        \item Реализовать поддержку RISC-V для клиента Instrew
              \begin{itemize}
                  \item Добавить процессорно-специфические патчи в minilibc и другие компоненты
                  \item Реализовать функции, связанные с RISC-V ABI
              \end{itemize}
        \item Выполнить тестирование
        \item Провести замеры производительности и сравнение их с аналогами
              % (например на SPEC CPU 2017)
    \end{enumerate}
\end{frame}

\begin{frame}
    \frametitle{План работы}
    %Тут уместно рассказать подробности по каждому пункту, но кратко.

    \textbf{Что уже сделано:}
    \begin{enumerate}
        \item Выполнен частичный обзор архитектуры Instrew и других ДБТ
        \item Реализованы некоторые процессорно-специфические компоненты
        \item Начата реализация ELF релокаций
    \end{enumerate}

    \textbf{Планируется к зимней защите:}
    \begin{enumerate}
        \setcounter{enumi}{3}
        \item Выполнить обзор
        \item Реализовать поддержку RISC-V для клиента Instrew
    \end{enumerate}

    \textbf{Планируется к защите ВКР:}
    \begin{enumerate}
        \setcounter{enumi}{5}
        \item Выполнить тестирование
        \item Провести замеры производительности и сравнение их с аналогами
    \end{enumerate}
\end{frame}

\end{document}
